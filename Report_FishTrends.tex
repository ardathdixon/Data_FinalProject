% Options for packages loaded elsewhere
\PassOptionsToPackage{unicode}{hyperref}
\PassOptionsToPackage{hyphens}{url}
%
\documentclass[
  12pt,
]{article}
\usepackage{lmodern}
\usepackage{amsmath}
\usepackage{ifxetex,ifluatex}
\ifnum 0\ifxetex 1\fi\ifluatex 1\fi=0 % if pdftex
  \usepackage[T1]{fontenc}
  \usepackage[utf8]{inputenc}
  \usepackage{textcomp} % provide euro and other symbols
  \usepackage{amssymb}
\else % if luatex or xetex
  \usepackage{unicode-math}
  \defaultfontfeatures{Scale=MatchLowercase}
  \defaultfontfeatures[\rmfamily]{Ligatures=TeX,Scale=1}
  \setmainfont[]{Times New Roman}
\fi
% Use upquote if available, for straight quotes in verbatim environments
\IfFileExists{upquote.sty}{\usepackage{upquote}}{}
\IfFileExists{microtype.sty}{% use microtype if available
  \usepackage[]{microtype}
  \UseMicrotypeSet[protrusion]{basicmath} % disable protrusion for tt fonts
}{}
\makeatletter
\@ifundefined{KOMAClassName}{% if non-KOMA class
  \IfFileExists{parskip.sty}{%
    \usepackage{parskip}
  }{% else
    \setlength{\parindent}{0pt}
    \setlength{\parskip}{6pt plus 2pt minus 1pt}}
}{% if KOMA class
  \KOMAoptions{parskip=half}}
\makeatother
\usepackage{xcolor}
\IfFileExists{xurl.sty}{\usepackage{xurl}}{} % add URL line breaks if available
\IfFileExists{bookmark.sty}{\usepackage{bookmark}}{\usepackage{hyperref}}
\hypersetup{
  pdftitle={What's the Catch? Recreational Fishing Trends in North Carolina (1990-2019)},
  pdfauthor={Ardath Dixon, Annie Harshbarger, Eva May},
  hidelinks,
  pdfcreator={LaTeX via pandoc}}
\urlstyle{same} % disable monospaced font for URLs
\usepackage[margin=2.54cm]{geometry}
\usepackage{graphicx}
\makeatletter
\def\maxwidth{\ifdim\Gin@nat@width>\linewidth\linewidth\else\Gin@nat@width\fi}
\def\maxheight{\ifdim\Gin@nat@height>\textheight\textheight\else\Gin@nat@height\fi}
\makeatother
% Scale images if necessary, so that they will not overflow the page
% margins by default, and it is still possible to overwrite the defaults
% using explicit options in \includegraphics[width, height, ...]{}
\setkeys{Gin}{width=\maxwidth,height=\maxheight,keepaspectratio}
% Set default figure placement to htbp
\makeatletter
\def\fps@figure{htbp}
\makeatother
\setlength{\emergencystretch}{3em} % prevent overfull lines
\providecommand{\tightlist}{%
  \setlength{\itemsep}{0pt}\setlength{\parskip}{0pt}}
\setcounter{secnumdepth}{5}
\usepackage{booktabs}
\usepackage{longtable}
\usepackage{array}
\usepackage{multirow}
\usepackage{wrapfig}
\usepackage{float}
\usepackage{colortbl}
\usepackage{pdflscape}
\usepackage{tabu}
\usepackage{threeparttable}
\usepackage{threeparttablex}
\usepackage[normalem]{ulem}
\usepackage{makecell}
\usepackage{xcolor}
\ifluatex
  \usepackage{selnolig}  % disable illegal ligatures
\fi

\title{What's the Catch? Recreational Fishing Trends in North Carolina
(1990-2019)}
\usepackage{etoolbox}
\makeatletter
\providecommand{\subtitle}[1]{% add subtitle to \maketitle
  \apptocmd{\@title}{\par {\large #1 \par}}{}{}
}
\makeatother
\subtitle{\url{https://github.com/ardathdixon/Data_FinalProject}}
\author{Ardath Dixon, Annie Harshbarger, Eva May}
\date{Spring 2021}

\begin{document}
\maketitle

\newpage
\tableofcontents 
\newpage
\listoftables 
\newpage
\listoffigures 
\newpage

\hypertarget{rationale-and-research-questions}{%
\section{Rationale and Research
Questions}\label{rationale-and-research-questions}}

As public awareness of increasing strains on ocean resources and
organisms grows (e.g.~see the reach of films such as Seaspiracy and
Sonic Sea), more attention is being placed on understanding fishing
patterns and the impacts they have on the oceans. Most of this attention
is placed on commercial and industrial fishing operations, which are
studied and managed by the federal agency NOAA, or the National Oceanic
and Atmospheric Administration. However, there are fewer studies on
recreational fishing, for which NOAA also aids in overseeing and
collects data. For this study, we chose to investigate recreational
fishing trends in North Carolina over a thirty-year period. The data,
whose origins are discussed more below, initially included 27 variables,
detailing information such as mode of fishing and wave (time period) the
data was caught in. For this analysis, we wanted to look specifically at
total catch during each wave, as we were running time series analyses
during the course of the project. Therefore, we focused on only one
catch estimation variable: total catch. We chose North Carolina due to
our familiarity with species here, and we chose two popular recreational
fishing species to investigate alongside all species combined. Trends in
recreational fishing data can give us information about human behavior,
species populations, and species movement patterns, which is why we
found this topic interesting and wanted to investigate it further. We
chose the following questions to guide our work:

\begin{enumerate}
\def\labelenumi{\arabic{enumi}.}
\item
  Are there trends in the amount of these fish caught over time? How do
  they compare?
\item
  What could these trends look like in the future?
\end{enumerate}

\newpage

\hypertarget{dataset-information}{%
\section{Dataset Information}\label{dataset-information}}

\hypertarget{data-retrieval}{%
\subsection{Data Retrieval:}\label{data-retrieval}}

For this analysis, we used data collected during Marine Recreational
Information Program (MRIP) surveys conducted by NOAA
(\href{https://www.fisheries.noaa.gov/recreational-fishing-data/about-marine-recreational-information-program}{\textbf{NOAA
n.d.}}). NOAA works with state and local partners to collect information
on the species and number of fish caught by fishers via in-person
communication, telephone surveys, and mail-in surveys. We retrieved the
data using the NOAA Recreational Fisheries Statistics Queries ``download
query'' tool (found
\href{https://www.fisheries.noaa.gov/data-tools/recreational-fisheries-statistics-queries}{\textbf{here}}).
We created three separate queries to download data: one for all species,
one for bluefish (\emph{Pomatomus saltatrix}), and one for black sea
bass (\emph{Centropristis striata}). For each query, we used the date
range 1990-2019 and requested catch estimate data by ``wave'', or
two-month period, for all waves, fishing modes, and areas of fishing for
the state of North Carolina. We downloaded the CSVs as ZIP files and
added them to our project repository.

All data and code for this project can be retrieved from the
\href{https://github.com/ardathdixon/Data_FinalProject}{\textbf{GitHub
repository}}.

\hypertarget{data-wrangling}{%
\subsection{Data Wrangling:}\label{data-wrangling}}

We began our analysis by converting waves to months, in order to process
the six annual waves using time series analyses. For NOAA fishing
records, wave 1 represents January and February, wave 2 represents March
and April, and this continues through the year. Therefore, we assigned
wave 1 catches to the date of January 1, wave 2 catches to March 1, and
beyond.

\hypertarget{more-on-wrangling-here}{%
\subsubsection{MORE ON WRANGLING HERE}\label{more-on-wrangling-here}}

\begin{table}[H]

\caption{\label{tab:table1}General Information About the Data Used}
\centering
\begin{tabular}[t]{l|l}
\hline
Detail & Description\\
\hline
Data Source & NOAA MRIP\\
\hline
Retrieved from & https://www.fisheries.noaa.gov/data-tools/recreational-fisheries-statistics-queries\\
\hline
Variables Used & Year, Wave, Total Catch (Number of fish), Mode, Area\\
\hline
Date Range & January 1990 - December 2019\\
\hline
\end{tabular}
\end{table}

\begin{table}[H]

\caption{\label{tab:table2}Total Catch Summaries (Number of Fish)}
\centering
\begin{tabular}[t]{l|r|r|r}
\hline
Summary Statistics & All Fish & Bluefish & Black Sea Bass\\
\hline
Minimum & 11869 & 26 & 1168\\
\hline
Mean & 12402954 & 1342064 & 411196\\
\hline
Median & 11292146 & 1064369 & 313437\\
\hline
Maximum & 34932698 & 5254124 & 1746847\\
\hline
\end{tabular}
\end{table}
\newpage

\hypertarget{exploratory-analysis}{%
\section{Exploratory Analysis}\label{exploratory-analysis}}

Following initial wrangling, we checked the number of waves without
catch records for each dataset by joining the existing data to a list of
all possible waves between Wave 1 of 1990 (represented by 1990-01-01)
and Wave 6 of 2019 (2019-11-01). The results of this exploration, which
informed our approach for interpolation, can be found in Table 3.

\begin{table}[H]

\caption{\label{tab:table3}Number of missing values from NOAA MRIP data}
\centering
\begin{tabular}[t]{l|r}
\hline
Dataset & Number of missing values\\
\hline
All fish & 11\\
\hline
Bluefish & 17\\
\hline
Black sea bass & 13\\
\hline
\end{tabular}
\end{table}

To fill the gaps with no data, we interpolated the likely values of
missing time periods. This linear interpolation incorporated the catch
numbers on either side chronologically. If Wave 1 of 1990 or wave 6 of
2019 was missing, we did not interpolate the value, as there would not
be a measurement available on both sides. We graphed the total catch
trends over time (with the newly interpolated values for missing
periods) as shown in Figure 1. With this visualization, we could compare
the three categories' recreational fishing catch patterns: all fish,
bluefish, and black sea bass.

\begin{figure}[H]

\hfill{}\includegraphics{Report_FishTrends_files/figure-latex/ggplot-1} 

\caption{Total Catch Patterns over Time}\label{fig:ggplot}
\end{figure}

\newpage

\hypertarget{analysis}{%
\section{Analysis}\label{analysis}}

\hypertarget{question-1-are-there-trends-in-the-amount-of-these-fish-caught-over-time-how-do-they-compare}{%
\subsection{Question 1: Are there trends in the amount of these fish
caught over time? How do they
compare?}\label{question-1-are-there-trends-in-the-amount-of-these-fish-caught-over-time-how-do-they-compare}}

After our exploratory analysis and interpolation of the missing data
points for each dataset, we created three time series for further
analysis. We investigated the trends in total catch for all fish (Figure
2), bluefish (Figure 3), and black sea bass (Figure 4) by decomposing
the time series into their seasonal, trend, and remainder components.
For all three time series, we observed a strong seasonal trend with low
catch totals in the winter and high catch totals in the summer.
Furthermore, each trend component showed an apparent increase in total
catch over time.

\begin{figure}[H]

\hfill{}\includegraphics{Report_FishTrends_files/figure-latex/All Fish Trends-1} 

\caption{Seasonal and Trend Decomposition for All Fish Total Catch}\label{fig:All Fish Trends}
\end{figure}

\begin{figure}[H]

\hfill{}\includegraphics{Report_FishTrends_files/figure-latex/Bluefish Trends-1} 

\caption{Seasonal and Trend Decomposition for Bluefish Total Catch}\label{fig:Bluefish Trends}
\end{figure}

\begin{figure}[H]

\hfill{}\includegraphics{Report_FishTrends_files/figure-latex/Black Sea Bass Trends-1} 

\caption{Seasonal and Trend Decomposition for Black Sea Bass Total Catch}\label{fig:Black Sea Bass Trends}
\end{figure}

\begin{table}[H]

\caption{\label{tab:table4}Seasonal Mann Kendall Tests}
\centering
\begin{tabular}[t]{l|r|l}
\hline
Fish Category & tau & 2-Sided P-value\\
\hline
All Fish & 0.4896552 & 0.000000e+00\\
\hline
Bluefish & 0.3235180 & 8.748902e-10\\
\hline
Black Sea Bass & 0.4095312 & 8.437695e-15\\
\hline
\end{tabular}
\end{table}

We ran a Seasonal Mann-Kendall test on each time series to test whether
there was a monotonic trend in the total number of fish caught over time
(Table 4). All three tests returned a statistically significant result
(All fish: tau = 0.49, p \textless{} 2.22 x 10\^{}-16; bluefish: tau =
0.32, p = 8.75 x 10\^{}-10; black sea bass: tau = 0.41, p = 8.44 x
10\^{}-15). Therefore, for all three time series, we reject the null
hypothesis that there is no monotonic trend in favor of the alternative
hypothesis that there is a trend in the data over time.

The strengths of these trends vary; catch for all fish has the strongest
trend while bluefish has the weakest trend. Nonetheless, catch for both
individual species and all species combined increases over the span of
the data, between the beginning of 1990 and the end of 2019.

\hypertarget{question-2-what-could-these-trends-look-like-in-the-future}{%
\subsection{Question 2: What could these trends look like in the
future?}\label{question-2-what-could-these-trends-look-like-in-the-future}}

To investigate future trends, we used forecasting in R to predict future
data based on the existing past data we pulled from NOAA. There are
several different methods through which to forecast, though only some of
them account for seasonality, which was necessary here due to the
seasonal components found in all three of our time series. We chose to
use the Holt-Winters forecasting method for this data. Holt-Winters is
more complex than simpler methods like naive but requires knowledge of
fewer additional variables that other models, like SARIMA, need in order
to run. Holt-Winters uses smooth exponentiating and varying weights of
past data - with more recent data weighed more - to predict future data.
Here, we predicted five years of data (where h = number of periods, and
each period = 2 months). In the resulting plots, the dark purple area
represents the 80\% confidence interval level for predicted data, while
the light purple area represents the 95\% confidence interval level.

\begin{figure}[H]

\hfill{}\includegraphics{Report_FishTrends_files/figure-latex/unnamed-chunk-1-1} 

\caption{Holt-Winters Catch Forecasts}\label{fig:unnamed-chunk-1}
\end{figure}

The Holt-Winters plots all show clear continued seasonality patterns and
trends. For Black Sea Bass and all species combined, overall future
trends are expected to be positive, like the past trends. For Bluefish,
the forecasted trend is slightly negative -- this is likely because the
most recent years of bluefish data do show a slight negative slope, even
though the trend is positive for the full twenty years in the dataset
(though as noted above, bluefish has the smallest tau, and therefore the
weakest trend). These forecasting plots are useful visualizations but
should not be considered fully accurate because of these inherent
complexities in fishing catch data that models like Holt-Winters cannot
account for.

\begin{quote}
\emph{Insert visualizations and text describing your main analyses.
Format your R chunks so that graphs are displayed but code and other
output is not displayed. Instead, describe the results of any
statistical tests in the main text (e.g., ``Variable x was significantly
different among y groups (ANOVA; df = 300, F = 5.55, p \textless{}
0.0001)''). Each paragraph, accompanied by one or more visualizations,
should describe the major findings and how they relate to the question
and hypotheses. Divide this section into subsections, one for each
research question.}
\end{quote}

\begin{quote}
\emph{Each figure should be accompanied by a caption, and each figure
should be referenced within the text}
\end{quote}

\newpage

\hypertarget{summary-and-conclusions}{%
\section{Summary and Conclusions}\label{summary-and-conclusions}}

\begin{quote}
\emph{Summarize your major findings from your analyses in a few
paragraphs. What conclusions do you draw from your findings? Relate your
findings back to the original research questions and rationale.}
\end{quote}

\hypertarget{strong-seasonal-trends}{%
\subsection{Strong seasonal trends}\label{strong-seasonal-trends}}

NOAA marine recreational fishing catch totals for North Carolina show
strong seasonal trends. Many more fish are caught in the summer, and
much fewer fish are caught in the winter, as demonstrated above (Figure
1). The high seasonality for all three datasets analyzed was confirmed
with the Seasonal Mann Kendall Tests, where all three P-values
\textless{} 0.05 (Table 4).

This seasonality is likely influenced by recreational fishing patterns,
where fishers are more likely to fish in the warm summer weather than
the cool winter weather. Another potential cause for the seasonal trends
is fish abundance and migration patterns, with higher populations of
fish in North Carolina waters during the summer than during the winter.
Total catch trends for all fish and Black Sea Bass showed unimodal peaks
and valleys overall, while Bluefish showed bimodal trends (Figure 1).
These bimodal Bluefish peaks could be due to their seasonal migration
patterns (\href{http://www.asmfc.org/species/bluefish}{ASMFC 2021}).

\hypertarget{overall-positive-trend}{%
\subsection{Overall positive trend}\label{overall-positive-trend}}

There was an increase in total catch of bluefish, black sea bass, and
all species combined over time. The driver of this increase in
recreational fishery landings is unknown, but it could be attributed to
increased fishing effort, with more fishers participating in
recreational fishing, or a change in methods or regulations that
resulted in individual fishers catching more fish.

\#\#I think it would be beneficial to add some references in this
section -- working on this! - Annie

Although the trend was generally positive for all three time series, it
was not uniform; in other words, the rate of change varied over time,
and there were brief periods where catch plateaued or decreased. This
variation could be caused by changes in recreational fishing regulations
over time, such as the increase or decrease of catch limits and size
limits, or temporary area closures. Furthermore, recreational fishing is
subject to all of the myriad factors that can influence behavior,
including but not limited to climate variation and current events. In a
recent example of how current events could impact recreational fishing
that is beyond the scope of our analysis, which extended only to the end
of 2019, the COVID-19 pandemic could have either increased recreational
fishing in 2020, when people were spending more time outside, or
decreased recreational fishing, as people were less able to travel to
the coast.

\hypertarget{limitations}{%
\subsection{Limitations}\label{limitations}}

The MRIP system works very well for its intended purposes, but it is
ultimately still a system based on estimation. Through MRIP, NOAA
interviews only some fishers, then uses mathematical modeling to
extrapolate on this collected survey data in order to create statewide
(or area-wide, depending on the dataset) estimates. While MRIP is the
best source of recreational fishing catch data, there is always room for
some error in its estimations.

Within each of our datasets, there were some missing values. These
values were not side by side or frequent, so we chose to linearly
(\emph{?}) interpolate our data in order to fill them in. This helped
our figures and analyses (e.g.~Mann-Kendall) to appear and run cleaner,
but interpolating has some drawbacks. While interpolations seemed to
follow the clear seasonal pattern in each dataset, most often the
interpolated data was in wave periods that were minimums in other years
of the datasets, so our interpolated values may have been a bit higher
than actual catch rates during those times. Interpolation is an
estimator for missing data, and it is important to acknowledge that our
interpolated data may not be representative of true values.

Finally, our forecasted data are much cleaner and less noisy than our
existing data. Noise in these datasets comes from external factors like
changes in fish catch limits or weather patterns, which this forecasting
method cannot take into account or predict. Our forecasting outputs are
therefore limited by the relatively simple methodology we chose to
employ. Holt-Winters remains a popular forecasting method and is
reasonable to use for our purposes here, but the inherent uncertainty in
this predicted data is important to acknowledge.

\hypertarget{future-recommendations}{%
\subsection{Future recommendations}\label{future-recommendations}}

\begin{itemize}
\tightlist
\item
  Comparisons of other species or other states
\item
  Catch per unit effort
\item
  Include earlier data
\end{itemize}

\newpage

\hypertarget{references}{%
\section{References}\label{references}}

\textless add references here if relevant, otherwise delete this
section\textgreater{}

National Oceanic and Atmospheric Administration. (n.d.). ``About the
Marine Recreational Information Program''. Retrieved from:
\url{https://www.fisheries.noaa.gov/recreational-fishing-data/about-marine-recreational-information-program},
accessed 25 Apr 2021.

\end{document}
